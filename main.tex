\documentclass[letterpaper,12pt,oneside]{book}
%\usepackage[a4paper,includeall,bindingoffset=0cm,margin=2cm,marginparsep=0cm,marginparwidth=0cm]{geometry}
\usepackage[DIV=14,BCOR=2mm,headinclude=true,footinclude=false]{typearea}
\usepackage[top=1in, left=0.9in, right=1.25in, bottom=1in]{geometry}
\usepackage{ThesisITL}
\usepackage{xcolor}
\usepackage[utf8]{inputenc}
\usepackage{url}
\usepackage[T1]{fontenc}
\usepackage[spanish,es-nodecimaldot,es-tabla]{babel}
\usepackage{graphicx}
\usepackage{tikz}
\usepackage{tocloft}
\graphicspath{{./Images/}}
\usepackage{setspace}
\usepackage{comment}
\usepackage{hyperref}
\urlstyle{same}
\hypersetup{
   colorlinks=true,
   urlcolor=blue,
   linkcolor=black,
   citecolor=black,
   filecolor=magenta,
   pdftitle={Sharelatex Example},
   pdfpagemode=FullScreen,
}

%%lorem ipsum text
\usepackage{lipsum}  

%% Remover sangría
\setlength\parindent{0pt}

%% Cambiar margenes
\usepackage{anysize} 			% Para personalizar el ancho de  los márgenes
\marginsize{3cm}{2.5cm}{2.5cm}{2.5cm} % Izquierda, derecha, arriba, abajo

%% Incluir encabezados
\usepackage{fancyhdr}
\fancypagestyle{plain}{\fancyhf{}\renewcommand{\headrulewidth}{0pt}} % To clear page numbers from footer, and header line at the start of every chapter

\pagestyle{fancy}
\fancyhf{}% Clear header/footer
\fancyhead[L]{\nouppercase\leftmark}
\fancyfoot[RO,RE]{\thepage}


% Incluir apéndices 
%\usepackage{appendix}
%\renewcommand{\appendixname}{Apéndices}
%\renewcommand{\appendixtocname}{Apéndices}
%\renewcommand{\appendixpagename}{Apéndices} 


%%% Include code 
\usepackage{listings}
\usepackage{color}
\definecolor{dkgreen}{rgb}{0,0.6,0}
\definecolor{gray}{rgb}{0.5,0.5,0.5}
\definecolor{mauve}{rgb}{0.58,0,0.82}

\lstset{frame=tb,
	language=C,
	aboveskip=3mm,
	belowskip=3mm,
	showstringspaces=false,
	columns=flexible,
	basicstyle={\small\ttfamily},
	numbers=none,
	numberstyle=\tiny\color{gray},
	keywordstyle=\color{blue},
	commentstyle=\color{dkgreen},
	stringstyle=\color{mauve},
	breaklines=true,
	breakatwhitespace=true,
	tabsize=3
}

\begin{document}
%%Presentación del documento	
\begin{titlepage}
  \thispagestyle{empty}
  \begin{minipage}[c][0.17\textheight][c]{0.25\textwidth}
    \begin{center}
    \hspace*{-13mm}
      \includegraphics[ height=3.5cm]{Images/itl-logo.png}
    \end{center}
\end{minipage}
\begin{minipage}[c][0.195\textheight][t]{0.75\textwidth}
    \begin{center}
    \vspace{0.3cm}
    {\color{black}\bfseries{\textsc{\large Tecnol\'ogico Nacional de M\'exico}} }\\[0.5cm]
    \vspace{0.3cm}
    {\color{blue}\hrule height4pt}
    \vspace{.1cm}
    {\color{olive}\hrule height3pt}
    \vspace{.8cm}
    \bfseries{\textsc{ \large Instituto Tecnol\'ogico de Le\'on}}\\[0cm] %
    \end{center}
\end{minipage}
\begin{minipage}[c][0.81\textheight][t]{0.25\textwidth}
   \vspace*{5mm}
   \begin{center}
    \hskip0.5mm
    \vspace{5mm}
    \hskip2pt
    {\color{blue}\vrule width4pt height13cm}
    \hskip0.1mm
    {\color{olive}\vrule width4pt height13cm} \\
    \hspace*{-5mm}
    \includegraphics[height=6.5cm]{Images/SEP.jpg}
    \end{center}
\end{minipage}
  \begin{minipage}[c][0.81\textheight][t]{0.75\textwidth}
    \begin{center}
      \vspace{1cm}

      {\color{black}{\large\scshape Informe Técnico de Residencia Profesional}}\\[.2in]

      \vspace{1cm}            

      \textsc{\LARGE Nombre descriptivo}\\[2.5cm] %T\hspace{0.5cm}E\hspace{0.5cm}S\hspace{0.5cm}I\hspace{0.5cm}S}\\[2.5cm]
%      \textsc{\large que para obtener el grado de:}\\[0.5cm]
    
      {\color{black}\textsc{\large Presenta:}}\\[0.5cm]
      \textsc{\large {Nombre Apellido Apellido }}\\[0.5cm]          
      {\color{black}\textsc{\large Número de control:}}\\[0.5cm]
	  \textsc{\large {xxxxxx}}\\[2.5cm]
%      \textsc{\large M\'etodos num\'ericos}\\[0.5cm]

%      \vspace{0.5cm}
      \flushleft{
      {\color{black}\textsc{\large Asesores:}}\\[0.5cm]
	  \textsc{\large{ Asesor interno: Dr. Alexis Torres-Carbajal}}\\[0.5cm]
	  \textsc{\large{ Asesor(a) externo: Titulo. Nombre completo}}\\[0.5cm]	}
%
%
%
%      {\large\scshape 
%        {\color{black}Profesor}\\[0.3cm] {Dr. Alexis Torres Carbajal}}\\[.2in]
%% Para dos asesores, comentar arriba y retirar comentario abajo          	
%        {\color{black}Profesor}\\[0.3cm] {Dr. Alexis Torres Carbajal \\ 
%          Dr. 2}}\\[.2in]

      \vspace{1.5cm}
       
      \large{Le\'on, Guanajuato \hspace*{3cm} \today}
    \end{center}
  \end{minipage}
\end{titlepage}
%---------------------------------
\chapter*{\centering Resumen}

Este apartado sintetiza: el objetivo del reporte (Introducción), los principales métodos empleados (Desarrollo), los resultados con mayor significancia logrados (Resultados) y las principales conclusiones.\\

En su redacción se debe cumplir con estos requisitos:
\begin{itemize}
	\item[1)] Consta de un solo párrafo
	\item[2)] No contiene citas bibliográficas 
	\item[3)] No contiene referencias a tablas o a figuras
	\item[4)] Se redacta en tiempo pasado
	\item[5)] No contiene siglas o abreviaturas, excepto lo que toda la audiencia conoce
\end{itemize}

\include{Portada/frontmater}
\chapter{Introducción}

Es el contenido global de lo que va a encontrarse en el trabajo. Incluye los aspectos relevantes de los antecedentes, la definición del problema, la justificación, los objetivos (generales y específicos) y la organización del documento. En este apartado se iniciará la paginación principal  con números arábigos (los capítulos podrán ir enumerados con romano o arábigos).

\lipsum[2-4]

\lipsum[2-4]

\lipsum[2-4]



%%Contenido del manuscrito
\chapter{Fundamentos}

El conocimiento existente acerca del tema que se desarrolla es fundamental abordarlo en este apartado. Los puntos a desarrollar son los siguientes: 

%Así se genera un listado
\begin{itemize}
	\item[1]El contexto en el que se desarrolla el trabajo
	\item[2]La teoría en la que se fundamenta
	\item[3]El estado del arte, que se refiere a la recopilación de toda la información relevante, que a la fecha exista, relacionada con la temática.
\end{itemize}


\lipsum[2-4]

\lipsum[2-4]

\lipsum[2-4]

\section{Nombre de la sección 1}
La aproximación teórica \cite{Bloomfield1935} se puede emplear para caracterizar problemas que involucran \cite{Traugott1990}...

\section{Nombre de la sección 2}
Así podemos incluir una figura en el manuscrito y hacer referencia a ella \ref{Fig:01}. El índice de figuras se crea en automático

\begin{figure}[htp!]
	\centering
	\includegraphics[width=7cm,height=5cm]{itl-logo.png}
	\caption{Pie de figura con un texto corto y descriptivo.}
	\label{Fig:01}
\end{figure}

\subsection{Nombre de la subsección 2.1}
Así podemos incluir código dentro del manuscrito:\\
\begin{lstlisting}
	// Hello.c
	#include <stdio.h>
	#include <math.h>
	
	int main(){
	float a;
	float b;
	float c;	
	int i;
	
	for(i=1;i<=100;i++){
		a=(float)i;
		b=pow((float)i,2);		
		c=a + b;
	}

	return 0;
    }
\end{lstlisting}


\chapter{Desarrollo (Metodología)}

En este apartado se detalla la experimentación que se llevó a cabo, en cada una de las etapas del trabajo de tesis y se debe de tener cuidado de ser consecuente. Si es necesario, apoyarse con un esquema a fin de mostrar de manera global el trabajo e ir presentando posteriormente la descripción de cada una de dichas etapas experimentales.   



\lipsum[2-4]

\lipsum[2-4]

\lipsum[2-4]
\chapter{Resultados y discusión}
Es la parte medular del reporte porque aquí se informa de los resultados obtenidos en cada etapa, siendo concordante con lo mencionado previamente en el apartado del Desarrollo. Se discuten además, los resultados, comparándolos con los datos obtenidos por otros investigadores, apoyándose  con referencias recientes. 

\lipsum[2-4]

\lipsum[2-4]

\lipsum[2-4]
\chapter{Conclusiones}

Concluir únicamente con los resultados obtenidos de la investigación. Es importante comentar que este apartado no se debe de considerar como un resumen de resultados.\\

Mencionar las recomendaciones que se consideren pertinentes, de acuerdo con las experiencias obtenidas, dentro del desarrollo del trabajo.


\lipsum[2-4]

\lipsum[2-4]

\lipsum[2-4]

%% Bibliografia
\bibliographystyle{IEEEtran}
\bibliography{References/predoc.bib}

\end{document}
