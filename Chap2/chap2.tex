\chapter{Fundamentos}

El conocimiento existente acerca del tema que se desarrolla es fundamental abordarlo en este apartado. Los puntos a desarrollar son los siguientes: 

%Así se genera un listado
\begin{itemize}
	\item[1]El contexto en el que se desarrolla el trabajo
	\item[2]La teoría en la que se fundamenta
	\item[3]El estado del arte, que se refiere a la recopilación de toda la información relevante, que a la fecha exista, relacionada con la temática.
\end{itemize}


\lipsum[2-4]

\lipsum[2-4]

\lipsum[2-4]

\section{Nombre de la sección 1}
La aproximación teórica \cite{Bloomfield1935} se puede emplear para caracterizar problemas que involucran \cite{Traugott1990}...

\section{Nombre de la sección 2}
Así podemos incluir una figura en el manuscrito y hacer referencia a ella \ref{Fig:01}. El índice de figuras se crea en automático

\begin{figure}[htp!]
	\centering
	\includegraphics[width=7cm,height=5cm]{itl-logo.png}
	\caption{Pie de figura con un texto corto y descriptivo.}
	\label{Fig:01}
\end{figure}

\subsection{Nombre de la subsección 2.1}
Así podemos incluir código dentro del manuscrito:\\
\begin{lstlisting}
	// Hello.c
	#include <stdio.h>
	#include <math.h>
	
	int main(){
	float a;
	float b;
	float c;	
	int i;
	
	for(i=1;i<=100;i++){
		a=(float)i;
		b=pow((float)i,2);		
		c=a + b;
	}

	return 0;
    }
\end{lstlisting}

